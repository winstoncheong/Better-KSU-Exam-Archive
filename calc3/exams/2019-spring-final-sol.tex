\documentclass[12pt,answers]{exam}
\usepackage{amsmath,amsfonts,amssymb,commath,mathtools,physics}
\usepackage{todonotes}
\newcommand{\vect}[1]{\left\langle #1 \right\rangle}
\newcommand{\tcurl}{\operatorname{curl}}

\pagestyle{headandfoot}
\firstpageheadrule
\runningheadrule
\firstpageheader{Math 222}{Final Exam|Solutions, Page \thepage\ of \numpages}{May 15, 2019}
\runningheader{Math 222}{Final Exam|Solutions, Page \thepage\ of \numpages}{May 15, 2019}
\runningfooter{}{}{}

\title{2019 Spring Calc 3 Final|Solutions}
\author{Winston Cheong}
\date{}

\begin{document}
% \maketitle

\begin{questions}
\question
Let $\vb{u} = \vect{1, 1, 0}$, $\vb{v} = \vect{1, 0, 1}$ and $\vb{w} = \vect{0, 1, 1}$.
\begin{parts}
\part[5] Compute the volume of the parallelopiped spanned by $\vb u$, $\vb v$ and $\vb{w}$. 
\begin{solution}
\[
    \mqty| 1 & 1 & 0 \\ 1 & 0 & 1 \\ 0 & 1 & 1| = 1 \cdot \mqty|0 & 1 \\ 1 & 1| - 1 \cdot \mqty|1 & 1 \\ 0 & 1| + 0 \cdot \mqty|1 & 0 \\ 0 & 1|
    = -1 -1 + 0 = -2
\]
So the volume is \boxed{2}
\end{solution}
\part[5] Compute the angle $\theta$ between $\vb v$ and $\vb w$.
\begin{solution}
    \[
        \cos \theta = \frac{\vb{u} \vdot \vb{w}}{\|\vb{v}\|\, \|\vb{w}\|} = \frac{1}{\sqrt 2\cdot \sqrt 2}
    \]
    Thus $\theta = \cos[-1](\frac12) = \boxed{\frac\pi3}$
\end{solution}
\part[5] Give the equation for the plane parallel to $\vb u$ and $\vb v$ and passing through the origin.
\begin{solution}
    The normal vector for the plane is 
    \[
    \vb{n} = \vb{u} \cross \vb{v} = \mqty| \vb{i} & \vb{j} & \vb{k} \\ 1 & 1 & 0 \\ 1 & 0 & 1| = \vect{1, -1, -1}
    \]
    Thus the equation for the plane is
    \[
        \boxed{\vect{1, -1, -1} \vdot \vect{x, y, z} = 0}
        \quad \text{or} \quad \boxed{x-y-z = 0}
    \]
\end{solution}
\end{parts}

\newpage
\question Consider the curve $\mathcal{C}$ given by the parametrization 
\[
    \vb{r}(t) = \vect{\sin(t), \cos(t), e^t}\quad \text{for } 0\le t \le \pi
\]
\begin{parts}
\part[5] Find the speed of $\vb{r}(t)$ as a function of $t$.
\begin{solution}
    \[
    s(t) = \|\vb{r}'(t)\| = \| \vect{\cos t, -\sin t, e^t} \| = \sqrt{\cos^2 t + \sin^2 t + e^{2t}}
    = \boxed{\sqrt{1 + e^{2t}}}
    \]
\end{solution}
\part[5] Compute the scalar line integral 
\[
    \int_{\mathcal C} 3x^2 z^2 + 3y^2 z^2 \dif s.
\]
\begin{solution}
    Formula sheet:
    \[
        \int_{\mathcal C} f(x, y, z) \dif s = \int_a^b f(\vb{r}(t)) \|\vb{r}'(t)\| \dif t
    \]
    So
    \begin{align*}
    \int_{\mathcal C} 3x^2 z^2 + 3y^2 z^2 \dif s 
    &= \int_0^\pi (3\sin^2 t e^{2t} + 3\cos^2 t e^{2t}) \sqrt{1+e^{2t}} \dif t \\ 
    &= \int_0^\pi 3e^{2t} \sqrt{1+e^{2t}} \dif t
    \qquad
    \text{u sub:}
    \left(
    \begin{aligned}
   u &= 1+e^{2t} \\ 
   \dif u &= 2e^{2t} \dif t \implies \dif t = \frac{\dif u}{2e^{2t}}
    \end{aligned} 
    \right)
    \\
    &= \int 3 e^{2t} \cdot u^{1/2} \frac{\dif u}{2e^{2t}} \\ 
    &= \frac32 \int u^{1/2} \dif u \\ 
    &= u^{3/2} = \left[(1+e^{2t})^{3/2}\right]_0^\pi
    = \boxed{(1+e^{2\pi})^{3/2} - 2^{3/2}}
    \end{align*}


\end{solution}
\end{parts}

\newpage
\question
Calculate the following quantities if they exist. Otherwise, explain why they do not exist. Justify either response.
\begin{parts}
\part[5] 
\[
    \lim_{(x, y) \to (0, 0)} \frac{x^2y^2}{x^2+y^2}
\]
\begin{solution}
    Converting to polar gives
    \begin{align*}
        \lim_{r\to0} \frac{r^4 \cos^2 \theta \sin^2 \theta}{r^2}
        &= \lim_{r\to0} r^2 \cos^2 \theta \sin^2 \theta = \boxed{0}
    \end{align*}
\end{solution}
\part[5]
For 
\[
    f(x, y, z) = \cosine(z^2 - y^2) + y \sin(x)
\]
compute 
\[
    f_{xy}(x,y,z)
\]
\begin{solution}
    \begin{align*}
        \frac{\partial^2}{\partial x \partial y} \cosine(z^2-y^2) &= 0 \\ 
        \frac{\partial^2}{\partial x \partial y} y \sin x &= 0 \\ 
        \shortintertext{thus}
        f_{xy} &= 0 + \cos x = \boxed{\cos x}
    \end{align*}
\end{solution}
\newpage
\part[5]
For $f(x,y,z) = x + y^2 + z^3$ find the change in $f(x,y,z)$ as one moves in the direction of the unit vector $\vb{u} = \frac{\sqrt 3}{3} \vect{1,1,1}$ starting at $\vect{2,0,-1}$.
\begin{solution}
    We know $D_{\vb u} f(p) = \nabla f_p \vdot \vb u$, so we compute
    \begin{align*}
        \nabla f &= \vect{1, 2y, 3z^2} \\ 
        \nabla f|_p &= \vect{1, 0, 3} \\ 
        \nabla f|_p \vdot \vb u &= \vect{1,0,3} \vdot \frac{1}{\sqrt 3} \vect{1,1,1} = \frac{1}{\sqrt 3}(1+0+3) = \boxed{\frac{4}{\sqrt 3}}
    \end{align*}

\end{solution}
\part[5] Find the equation for the tangent plane to the surface 
\[
    z = xy
\]
at the point $(1,2,2)$.
\begin{solution}
    The tangent plane at a point $p = (a,b)$ is given by 
    \[
        z = f(p) + f_x(p)(x-a) + f_y(p)(y-b)
    \]
    so we compute
    \begin{alignat*}{2}
        f_x &= y & f_y &= x\\ 
        f_x(p) &= 2 &\qquad f_y(p) &= 1
    \end{alignat*}
    giving
    \begin{equation*}
        \boxed{z = 2 + 2(x-1) + 1(y-2)} \text{ or } \boxed{2x+y-z = 2}
    \end{equation*}
\end{solution}
\end{parts}

\newpage
\question
Let 
\[
    f(x,y) = x^3-12x+y^2
\]
and $\mathcal{D}$ be the square $[-3,3]\times[-3,3]$.
\begin{parts}
\part[5] Find the critical points of $f(x, y)$ in the interior of $\mathcal{D}$.
\begin{solution}
    Solving $\nabla f = \vect{3x^2 - 12, 2y} = \vect{0, 0}$ gives $x = \pm 2$ and $y = 0$. So the critical points for $f$ are 
    \boxed{(2, 0), (-2, 0)}
\end{solution}
\part[5] Describe the local behavior of $f(x, y)$ at the critical points found in part (a).
\begin{solution}
    The discriminant is 
    \[
        D = f_{xx} f_{yy} - (f_{xy})^2 = 12x
    \]
    Evaluating the discriminant at the previously found critical points:
    \begin{align*}
        D_{(2, 0)} &= 24 > 0 \quad \text{and}\quad f_{xx}(2, 0) = 12 > 0 \quad \text{ so } \boxed{(2, 0) \text{ is a local min}} \\ 
        D_{(-2, 0)} &= -24 < 0 \quad \text{so } \boxed{(-2, 0) \text{ is a saddle point}}
    \end{align*}
\end{solution}

\newpage
\part[5] Find the maximum value of $f$ on $\mathcal D$.
\begin{solution}
    Let $\gamma_1, \gamma_2, \gamma_3, \gamma_4$ denote paths along the right, left, top, and bottom sides of the square $\mathcal{D}$, respectively.

    Along $\gamma_1$, $f(3, y) = -9 + y^2$, with $y \in [-3, 3]$. 
    Looking for critical points along $\gamma_1$, we compute
    $f' = 2y = 0 \implies y = 0$, so $(3, 0)$ is a critical point along $\gamma_1$. We test on the critical points and the end points of this path and obtain
    \[
        f(3, 0) = -9 \qquad f(3, -3) = 0 \qquad f(3, 3) = 0
    \]
    Along $\gamma_2$, $f(-3, y) = 9 + y^2$, with $y \in [-3, 3]$. 
    Looking for critical points along $\gamma_2$, we compute $f' = 2y = 0 \implies y = 0$, so $(-3, 0)$ is a critical point along $\gamma_2$. 
    We test on the critical points and the end points of this path and obtain
    \[
        f(-3, 0) = 9 \qquad f(-3, -3) = 18 \qquad f(-3, 3) = 18
    \]
    Along $\gamma_3$, $f(x, 3) = x^3 - 12x + 9$, with $x \in [-3, 3]$.
    Looking for critical points along $\gamma_3$, we compute $f' = 3x^2 - 12 = 0 \implies x^2 = 4 \implies = \pm 2$. So $(\pm2, 3)$ are critical points along this path.
    We test on the critical points and the end points of this path and obtain
    \[
        f(2, 3) = -7 \qquad f(-2, 3) = 25 \qquad f(-3, 3) = 18 \qquad f(3,3) = 0
    \]
    Finally, along $\gamma_4$, $f(x, -3) = x^3 - 12x + 9$, with $x \in [-3, 3]$.
    This yields critical points $(\pm2, -3)$. As $f(x, y)$ will not distinguish the difference between $y=3$ and $y=-3$, the values for $\gamma_4$ are the same as for $\gamma_3$:
    \[
        f(2, -3) = -7 \qquad f(-2, -3) = 25 \qquad f(-3, -3) = 18 \qquad f(3,-3) = 0
    \]
    Thus the maxiumum value for $f$ on $\mathcal D$ is \boxed{25} (which occurs at $(-2, \pm 3)$).
\end{solution}
\end{parts}

\newpage
\question[10]
Let $\mathcal W = [0, 1] \times [-1, 0] \times [0, 2]$. 
Evaluate the triple integral
\[
    \iiint_{\mathcal W} (2x + z)e^y \dif V
\]
\begin{solution}
    \begin{align*}
        \int_0^2 \int_{-1}^0 \int_0^1 (2x+z)e^y \dif x \dif y \dif z
        &=  \int_{-1}^0 e^y \dif y \cdot \int_0^2 \int_0^1 (2x+z) \dif x \dif z \\ 
        &= \eval{e^y}_{-1}^0 \cdot \int_0^2 \left[x^2 + xz\right]_{x=0}^1 \dif z \\ 
        &= \left(1-\frac 1e\right) \cdot \int_0^2 (1+z) \dif z \\ 
        &= \left(1-\frac 1e\right) \cdot \left[z + \frac12 z^2\right]_0^2 \\ 
        &= \boxed{4\left(1 - \frac1e\right)}
    \end{align*}
\end{solution}

\newpage
\question Evaluate the following integrals.
\begin{parts}
\part[10] Let $\mathcal D$ be the region $x^2 + y^2 \le 4$, $0 \le y$, $x \le 0$. Evaluate
\[
    \iint_{\mathcal D} 3x \dif A.
\]
\begin{solution}
The region is a quarter circle of radius 2 in the second quadrant.
We convert to polar and integrate:
\begin{align*}
    \iint_{\mathcal D} 3x \dif A
    &= \int_{\pi/2}^\pi \int_0^2 3r\cos \theta \ r \dif r \dif \theta \\ 
    &= \int_0^2 3r^2 \dif r \cdot \int_{\pi/2}^\pi \cos \theta \dif \theta \\ 
    &= \eval{r^3}_0^2 \cdot \eval{\sin \theta}_{\pi/2}^\pi \\ 
    &= (8-0) \cdot (0-1) \\ 
    &= \boxed{-8}
\end{align*}
\end{solution}
\part[10] Let $\mathcal D$ be the region between the lines $y = -x$, $y = -1$ and $x = -1$.
Compute the integral 
\[
    \iint_{\mathcal D} 2y \dif A.
\]
\begin{solution}
    The integral to set up is clear from a sketch of the region $\mathcal D$.
    \begin{align*}
    \iint_{\mathcal D} 2y \dif A
    &= \int_{-1}^1 \int_{-1}^{-x} 2y \dif y \dif x \\ 
    &= \int_{-1}^1 y^2\Bigr]_{-1}^{-x} \dif x \\ 
    &= \int_{-1}^1 x^2 - 1 \dif x \\ 
    &= \eval{\frac{x^3}{3} - x}_{-1}^1 = \boxed{-\frac43}
    \end{align*}
\end{solution}
\end{parts}

\newpage
\question Let 
\[
    \vb F = \vect{2x+yz, xz, xy}.
\]
\begin{parts}
\part[5] If $\vb F$ is a conservative vector field, find a potential. Otherwise, explain why it is not conservative.
\begin{solution}
    $\vb{F}$ is a conservative vector field, with potential function 
    $f = x^2 + xyz$.
    One can check that $\nabla f = \vb{F}$.
\end{solution}
\part[5] Let $\mathcal C$ be the oriented curve with parametrization
\[
    \vb r(t) = \vect{\sin[6](\pi t)+t+1, e^t + e^{-t}, e^{t^2-1} - 1}
\]
for $-1 \le t \le 1$.
Compute 
\[
    \int_{\mathcal C} \vb F \vdot \dif \vb r.
\]
\begin{solution}
Since $\vb{F}$ is conservative, 
\begin{align*}
\int_{\mathcal C} \vb F \vdot \dif \vb{r} 
&= f(\vb{r}(1)) - f(\vb{r}(-1))  \\ 
&= f(2, e + e^{-1} , 0) - f(0, e + e^{-1}, 0) \\ 
&= \boxed{4}
\end{align*}
\end{solution}

\newpage
\part[5] Is there a vector potential for $\vb{F}$ (a vector field $\vb{A}$ that satisfies $\vb{F} = \operatorname{curl}(\vb{A})$)? 
Explain your response.
\begin{solution}
  No. If $\vb{F}$ had a vector potential, then it would necessarily follow that $\operatorname{div} \vb{F} = 0$. However, 
  \[
    \operatorname{div} \vb{F} = 2 + 0 + 0 = 2 \ne 0
  \]
\end{solution}

\part[5] Let $\mathcal S$ be the sphere $x^2 + y^2 + z^2 = 9$ oriented outwardly.
Compute the surface integral
\[
    \iint_{\mathcal S} \vb{F} \vdot \dif \vb{S}.
\]
State any theorems used in the computation.
\begin{solution}
    By the divergence theorem, we know
    \[
    \iint_{\mathcal S} \vb{F} \vdot \dif \vb{S} 
    = \iiint_{\mathcal E} \operatorname{div} \vb F \dif V
    \]
    where $\mathcal E$ is the region for which $\partial \mathcal E = \mathcal S$.
    For this problem, $\mathcal E$ is the ball of radius 3 centered at the origin.
    Since $\operatorname{div} F = P_x + Q_y + R_z = 2 + 0 + 0 = 2$,
    we have 
    \begin{align*}
    \iint_{\mathcal S} \vb{F} \vdot \dif \vb{S} 
    &= \iiint_{\mathcal E} \operatorname{div} \vb F \dif V \\ 
    &= \iiint_{\mathcal E} 2 \dif V \\ 
    &= 2 \cdot \frac43 \pi \cdot 3^2 = \boxed{24 \pi}
    \end{align*}
\end{solution}
\end{parts}

\newpage
\question
Let $\mathcal D$ be the lower half disc
\[
    \mathcal D = \{(x, y) : x^2 + y^2 \le 1,\, y \le 0\}.
\]
The boundary of $\mathcal D$ consists of the line segment $\mathcal C_1$ along the $x$-axis oriented from $(1,0)$ to $(-1, 0)$ and the semi-circle 
\[
    \mathcal C_2 = \{(x, y) : y = -\sqrt{1-x^2},\, -1 \le x \le 1\}
\]
oriented counter-clockwise. Let $\vb F$ be the vector field 
\[
    \vb{F} = \vect{-yx^2, xy^2}.
\]
\begin{parts}
\part[5]
Using polar coordinates, calculate the double integral
\[
    \iint_{\mathcal D} x^2 + y^2 \dif A
\]
\begin{solution}
    \begin{align*}
        = \int_\pi^{2\pi} \int_0^1 r^2 \cdot r \dif r \dif \theta 
        &= \int_\pi^{2\pi} 1 \dif \theta \cdot \int_0^1 r^3 \dif r \\ 
        &= \pi \cdot \left[\frac{r^4}{4}\right]_0^1 = \boxed{\frac\pi4}
    \end{align*}
\end{solution}

\newpage
\part[5] Compute the line integral
\[
    \int_{\mathcal C_1} \vb{F} \vdot \dif \vb{r}
\]
\begin{solution}
    We konw that
\[
    \int_{\mathcal C_1} \vb{F} \vdot \dif \vb{r} = \int_a^b \vb F(\vb r(t)) \vdot \vb r'(t) \dif t
\]

    The curve $\mathcal C_1$ can be parametrized by $\vb r(t) = \vect{1-2t,0}$ where $t \in [0, 1]$.
    Computing $\vb F(\vb r(t)) = \vect{0,0}$, we get
\[
    \int_{\mathcal C_1} \vb{F} \vdot \dif \vb{r} 
    = \int_0^1 0 \dif t = \boxed{0}
\]
\end{solution}
\part[5] Using only Green's Theorem and the computations in parts (a) and (b), compute the vector line integral 
\[
    \int_{\mathcal C_2} \vb{F} \vdot \dif \vb{r}
\]
\begin{solution}
    Let $\vb F = \vect{P,Q}$.
    By Green's theorem,
    \begin{align*}
        \iint_{\mathcal D} (Q_x - P_y) \dif A
        &= \int_{\mathcal C} \vb F \vdot \dif \vb r \\ 
        &= \int_{\mathcal C_1} \vb F \vdot \dif \vb r 
        + \int_{\mathcal C_2} \vb F \vdot \dif \vb r .
    \end{align*}
    For this problem, we have 
    $Q_x - P_y = y^2 + x^2$, which gives 
    \[
        \iint_{\mathcal D} (y^2 + x^2) \dif A
        = \int_{\mathcal C_1} \vb F \vdot \dif \vb r 
        + \int_{\mathcal C_2} \vb F \vdot \dif \vb r .
    \]
    Substituting in the results from parts (a) and (b), 
    \[
        \frac\pi4 = 0 + \int_{\mathcal C_2}\vb F \vdot \dif \vb r .
    \]
    Hence
    \[
    \int_{\mathcal C_2} \vb{F} \vdot \dif \vb{r} = \boxed{\frac\pi4}
    \]
\end{solution}
\end{parts}

\newpage
\question
Let $\mathcal S$ be the cylinder
\[
    \{(x, y, z) : x^2 + y^2 = 1,\, 0 \le z \le 3\}
\]
oriented outward and $\vb{F} = \vect{zy, -zx, 0}$.
\begin{parts}
\part[5] Compute $\tcurl(\vb F)$.
\begin{solution}
    \begin{align*}
        \tcurl(\vb F) = 
        \mqty| \vb i & \vb j & \vb k \\ 
        \partial_x & \partial_y & \partial_z \\ 
        P & Q & R|
        &= \vect{R_y - Q_z, P_z - R_x, Q_x - P_y} \\ 
        &= \vect{0 + x , y - 0, -z - z}
        = \boxed{\vect{x, y, -2z}}
    \end{align*}
\end{solution}
\part[5] Calculate
\[
    \iint_{\mathcal S} \tcurl(\vb{F}) \vdot \dif \vb{S}
\]
\begin{solution}
    \todo[inline]{figure out..messed up}
\end{solution}
\newpage
\part[5] The boundary of $\mathcal S$ consists of a unit circle $\mathcal C_1$ on the $xy$-plane oriented counterclockwise and a unit circle $\mathcal C_2$ on the $z=3$ plane oriented clockwise.
Noting that the vector field is zero on the $xy$-plane, one easily sees that 
\[
    \int_{\mathcal C_1} \vb{F} \vdot \dif \vb{r} = 0
\]
Using only this fact, Stokes' Theorem and your result from part (b), compute the vector line integral
\[
    \int_{\mathcal C_2} \vb{F} \vdot \dif \vb{r}
\]
\begin{solution}
    \todo[inline]{figure out..messed up}
\end{solution}
\end{parts}
\end{questions}
\end{document}
