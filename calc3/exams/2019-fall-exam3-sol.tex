\documentclass[12pt,answers]{exam}

\usepackage{amsmath,amsfonts,amssymb,mathtools,physics,commath}
\usepackage{cancel}
\newcommand{\vect}[1]{\left\langle #1\right\rangle}

\pagestyle{headandfoot}
\firstpageheadrule
\runningheadrule
\firstpageheader{Math 222}{Exam 3|Solutions, Page \thepage\ of \numpages}{November 21, 2019}
\runningheader{Math 222}{Exam 3|Solutions, Page \thepage\ of \numpages}{November 21, 2019}
\runningfooter{}{}{}

\title{2019 Fall Calc 3 Exam 3|Solutions}
\author{Winston Cheong}
\date{}

\begin{document}
% \maketitle
\begin{questions}
	\question[15]
	Calculate the integral
	\[
		\iiint_{\mathcal B} x^2 y \cos(xyz) \dif V
	\]
	where $\mathcal B = [0,\pi] \times [0,1] \times [-1,0]$.
	\begin{solution}
		\begin{align*}
			\MoveEqLeft \int_0^\pi \int_0^1 \int_{-1}^0 x^2 y \cos(xyz) \dif z \dif y \dif x \\ 
			&= 
			\int_0^\pi \int_0^1 x \bigl[\sin(xyz)\bigr]_{z=-1}^0 \dif y \dif x \\ 
			&= 
			\int_0^\pi \int_0^1 x \sin(xy) \dif y \dif x \\ 
			&= 
			\int_0^\pi - \bigl[\cos(xy)\bigr]_{y=0}^1 \dif x \\ 
			&= - \int_0^\pi \cos(x) - 1 \dif x \\
			&= - \bigl[ \sin(x) - x \bigr]_0^\pi
			= - (-\pi - 0) = \boxed{\pi}
		\end{align*}
	\end{solution}

	\newpage
	\question[15]
	Calculate the integral of
	\[
		f(x,y) = (1-x)^2
	\]
	over the region
	\[
		\mathcal D : 0 \le x \le 1-y^2, \quad 0 \le y
	\]
	\begin{solution}
		The region is the upper half of a sideways parabola.
		\begin{align*}
			\int_0^1 \int_0^{1-y^2} (1-x)^2 \dif x \dif y
			&= \int_0^1 \int_0^{1-y^2} 1 - 2x + x^2 \dif x \dif y\\
			&= \int_0^1 \bigl[x - x^2 + \frac13 x^3\bigr]_0^{1-y^2} \dif y\\
			&= \int_0^1 1-y^2 - (1-y^2)^2 + \frac13 (1-y^2)^3 \dif y \\ 
			&= \int_0^1 1-y^2 - (1-2y^2 + y^4) + \frac13 (1- 3y^2 + 3y^4 - y^6) \dif y \\ 
			&= \int_0^1 \frac13 - \frac13 y^6 \dif y\\
			&= \frac13 \left[ y - \frac17 y^7 \right]_0^1 
			= \frac13 \left(1-\frac17\right) = \boxed{\frac27}
		\end{align*}
	\end{solution}

	\newpage
	\question Consider the region
	\[
		\mathcal W : x^2 + y^2 + z^2 \le 25, \quad x^2 + y^2 \ge 16
	\]
	\begin{parts}
		\part[10]
		Express the volume $\mathcal W$ as an iterated integral using cylindrical coordinates.
		\begin{solution}
			The bounds can be expressed as 
			\[
				16 \le r^2 \le 25 - z^2 \implies 4 \le r \le \sqrt{25-z^2}
			\]
			This implies that $z$ ranges from $-3$ to 3.
			Hence the volume is calculated by the integral
			\[
				\boxed{ \int_{-3}^3 \int_4^{\sqrt{25-z^2}} \int_0^{2\pi} 1 \cdot r \dif \theta \dif r \dif z }
			\]
		\end{solution}

		\part[5] Evaluate the integral to obtain $\text{Vol}(\mathcal W)$.
		\begin{solution}
			\begin{align*}
				\MoveEqLeft \int_{-3}^3 \int_4^{\sqrt{25-z^2}} \int_0^{2\pi} 1 \cdot r \dif \theta \dif r \dif z \\ 
				&= 
				2\pi \int_{-3}^3 \int_4^{\sqrt{25-z^2}} r \dif r \dif z\\ 
				&=
				\pi \int_{-3}^3 \left[ r^2 \right]_4^{\sqrt{25-z^2}} \dif z\\ 
				&= \pi \int_{-3}^3 (9-z^2) \dif r \dif z \\ 
				&= \pi \left[ 9z - \frac13 z^3 \right]_{-3}^3 \\ 
				&= \pi (27 - 9 - (-27 + 9)) = \boxed{36 \pi}
			\end{align*}
		\end{solution}
	\end{parts}

	\newpage
	\question
	Let $\mathcal D$ be the parallelogram in the plane with vertices $(0, 0), (1, 0), (2, 3), (3, 3)$.
	\begin{parts}
		\part[10] Find a linear map $G(u,v)$ which sends the unit square $[0,1] \times [0,1]$ to $\mathcal D$.
		\begin{solution}
			The desired mapping is
			\begin{align*}
				(0,0) &\mapsto (0,0) \\
				(1,0) &\mapsto (1,0) \\
				(0,1) &\mapsto (2,3) \\
				(1,1) &\mapsto (3,3) \\
			\end{align*}
			hence is
			\[
				G(u,v) = (u+2v, 3v)
			\]
		\end{solution}

		\part[5] Compute the Jacobian of $G$.
		\begin{solution}
			\[
				\text{Jac}(G) = \dpd xu \dpd yv - \dpd xv \dpd yu = 1 \cdot 3 - 2 \cdot 0 = \boxed{3}
			\]
		\end{solution}

		\part[5] Use the change of variables formula to compute the integral
		\[
			\iint_{\mathcal D} e^{3x-2y} \dif A
		\]
		\begin{solution}
			\begin{align*}
				\iint_{\mathcal D} e^{3x-2y} \dif A
				&= \int_0^1 \int_0^1 e^{3(u+2v) - 2(3v)} |3| \dif u \dif v \\
				&= \int_0^1 \int_0^1 3e^{3u} \dif u \dif v \\ 
				&= \int_0^1 \dif v \cdot \int_0^1 3e^{3u} \dif u \\ 
				&= \bigl[e^{3u}\bigr]_0^1 = \boxed{e^3 - 1}
			\end{align*}
		\end{solution}
	\end{parts}

	\newpage
	\question[15] Calculate
	\[
		\int_{\mathcal C} e^{x^2+y^2+z^2} \dif s
	\]
	where $\mathcal C$ is the equator of a sphere, centered at the origin, of radius 3.
	\begin{solution}
		The path has the parameterization
		\[
			\vb{r}(t) = \vect{3\cos t, 3\sin t, 0} \quad t \in [0, 2\pi]
		\]
		We compute
		\begin{align*}
			\vb{r}'(t) &= \vect{-3\sin t, 3 \cos t, 0} \\
			\|\vb{r}'\| &= \sqrt{9\sin^2 t + 9 \cos^2 t} = 3
		\end{align*}
		so
		\begin{align*}
			\int_{\mathcal C} e^{x^2+y^2+z^2} \dif s
			&= \int_a^b f(\vb{r}(t)) \| \vb{r}'(t)\| \dif t \\
			&= \int_0^{2\pi} e^9  \cdot 3 \dif t
			= \boxed{6\pi e^9}
		\end{align*}
	\end{solution}

	\newpage
	\question[10]
	Evaluate
	\[
		\int_{\mathcal C} \vb{F} \vdot \dif \vb{r}
	\]
	where $\vb F(x,y,z) = \vect{z, xyz, x}$ and $\mathcal C$ is the curve parameterized by 
	\[
		\vb{r}(t) = (e^t, t, e^{-t})
	\]
	for $0 \le t \le 2$.
	\begin{solution}
		\begin{align*}
			\int_{\mathcal C} \vb{F} \vdot \dif \vb{r} 
			&= \int_a^b \vb{F}(\vb{r}(t)) \vdot \vb{r}'(t) \dif t \\
			&= \int_0^2 \vect{e^{-t}, t, e^t} \vdot \vect{e^t, 1, -e^{-t}} \dif t \\
			&= \int_0^2 1 + t - 1 \dif t
			= \left[\frac{t^2}{2}\right]_0^2
			= \boxed{2}
		\end{align*}
	\end{solution}

	\newpage
	\question
	Consider the vector field
	\[
		\vb{F}(x,y,z) = \vect{y+z, x+z, x+y}
	\]
	\begin{parts}
		\part[5] Does $\vb{F}$ satisfy the cross-partials condition? Verify your response.
		\begin{solution}
			Yes.
			Denoting the component functions $\vb{F} = \vect{P,Q,R}$, we check that
			\begin{itemize}
				\item $R_y = Q_z$: $1 = 1$ \checkmark
				\item $P_z = R_x$: $1 = 1$ \checkmark
				\item $Q_x = P_y$: $1 = 1$ \checkmark
			\end{itemize}
		\end{solution}

		\part[5] Find a potential for $\vb{F}$ if one exists. If not, explain why.
		\begin{solution}
			\[
				xy + xz + yz
			\]
		\end{solution}
	\end{parts}
\end{questions}
\end{document}
