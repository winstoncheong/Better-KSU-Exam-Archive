\documentclass[12pt,answers]{exam}

\usepackage{amsmath,amsfonts,amssymb,mathtools,physics,commath}
\usepackage{enumitem}
\usepackage{cancel}
\usepackage{todonotes}
\usepackage{outlines}
\newcommand{\vect}[1]{\left\langle #1\right\rangle}

\pagestyle{headandfoot}
\firstpageheadrule
\runningheadrule
\firstpageheader{Math 222}{Exam 2 Make Up|Solutions, Page \thepage\ of \numpages}{October 18, 2018}
\runningheader{Math 222}{Exam 2 Make Up|Solutions, Page \thepage\ of \numpages}{October 18, 2018}
\runningfooter{}{}{}

\title{2018 Fall Calc 3 Exam 2 Make Up|Solutions}
\author{Winston Cheong}
\date{}

\begin{document}
% \maketitle
\begin{questions}
	\question[16]
	Computation
	\begin{parts}
		\part
		Let $f(x, y) = (2x^2 + 3xy + y^2)\exp(x^3)$. Find all of the first partial derivatives. (In case you haven't seen it before, ``$\exp(u)$'' is the same thing as $e^u$.)
		\begin{solution}
			\begin{align*}
				f_x &= (4x + 3y)\exp(x^3) + (2x^2 + 3xy + y^2)\exp(x^3)(3x^2) \\ 
				f_y &= \exp(x^3)(3x+2y)
			\end{align*}
		\end{solution}

		\part
		Let $g(x,y) = \dfrac{x^2}{\sqrt{2x^2 + y^2}}$. Find the first partial derivative with respect to $x$ and simplify it.
		\begin{solution}
			\begin{align*}
				g_x &= \frac{\sqrt{2x^2+y^2} \cdot 2x - x^2 \cdot \frac{1}{2\sqrt{2x^2+y^2}}\cdot 4x}{2x^2+y^2} \\ 
				&= \frac{(2x^2 + y^2)(2x) - 2x^3}{(2x^2+y^2)^{3/2}} \\ 
				&= \boxed{\frac{2x^3 + 2xy^2}{(2x^2+y^2)^{3/2}}}
			\end{align*}
		\end{solution}
	\end{parts}

	\newpage
	\question[12]
	A certain differentiable function satisfies:
	\begin{enumerate}[label=(\alph*)]
		\item $f(2,5) = -7$, and $f(-1, 4) = \pi$.
		\item $\nabla f(2,5) = (-8,9)$, and $\nabla f(-1, 4) = (\sqrt 6, e^{-2})$.
	\end{enumerate}
	At each of the two points in question (i.e.~at $(2,5)$ and at $(-1, 4)$) answer the following questions:
	\begin{parts}
		\part In what direction is the function increasing the fastest and what is the rate of change in that direction?
		\begin{solution}~\\
			At $(2, 5)$, the function is increasing the fastest in the direction $\nabla f(2, 5) = \vect{-8, 9}$, with rate of change $\|\vect{-8,9}\| = \sqrt{64+81} = \sqrt{145}$.

			At $(-1, 4)$, the function is increasing the fastest in the direction $\nabla f(-1, 4) = \vect{\sqrt 6, e^{-2}}$, with rate of change $\|\vect{\sqrt 6, e^{-2}}\| = \sqrt{6 + e^{-4}}$.
		\end{solution}

		\part What is the directional derivative in the direction of the vector $\vect{6,-8}$?
		\begin{solution}
			The unit vector in the direction $\vect{6, -8}$ is $\vect{3/5, -4/5}$.
			So
			\begin{align*}
				D_{\vect{3/5,-4/5}} f(2,5) 
				&= \nabla f(2,5) \vdot \vect{3,-4} \cdot \frac15 \\ 
				&= \vect{-8, 9} \vdot \vect{3,-4} \cdot \frac15 \\
				&= (-24-36) \cdot \frac15
				= \boxed{-12}
				\shortintertext{and}
				D_{\vect{3/5,-4/5}} f(-1,4) 
				&= \nabla f(-1,4) \vdot \vect{3,-4} \cdot \frac15 \\ 
				&= \vect{\sqrt6, e^{-2}} \vdot \vect{3,-4} \cdot \frac15 \\
				&= \boxed{(3 \sqrt 6 - 4 e^{-2}) \cdot \frac15}
			\end{align*}
		\end{solution}

		\part What is the tangent plane and/or the linear approximation at each of the two points?
		\begin{solution}
			\begin{align*}
				(2, 5) &:\quad z = -7 + -8(x-2) + 9(y-5) \\ 
				(-1, 4) &:\quad z = \pi + \sqrt 6 (x+1) + e^{-2} (y-4)
			\end{align*}
		\end{solution}
	\end{parts}

	\newpage
	\question[12]
	Set up \textbf{but do not solve} the following problems. As part of setting these problems up, you should list the unknowns and the equations that you would need to use to find them. You \textbf{should also do} all of the \textbf{derivative} calculations, but the \textbf{algebra} is totally unmanageable, so do \textbf{not} attempt it!
	\begin{parts}
		\part Maximize $f(x, y) = x^2 \cos(2y)$ \\ 
		Subject to $g(x,y) = x^4 + y^6 = 2$.
		\begin{solution}
			\begin{align*}
				\nabla f &= \lambda \nabla g \\ 
				\vect{2x \cos(2y), -2x^2 \sin(2y)}  &= \lambda \vect{4x^3, 6y^5}
			\end{align*}
			The system to solve is:
			\[
				\left\{
					\begin{aligned}
						2x \cos(2y) &= 4\lambda x^3 \\ 
						-2x^2 \sin(2y) &= 6\lambda y^5 \\
						x^4 + y^6 &= 2
					\end{aligned}
				\right.
			\]
		\end{solution}

		\part Maximize $F(x,y,z) = \cos(xy^2 + yz^2 + zx^2)$\\
		Subject to $G(x,y,z) = 2x + 3y + 4z = 0$\\
		and $H(x,y,z) = x^4 + z^4 = 625$.
		\begin{solution}
			\begin{align*}
				\nabla F &= 
				-\sin(xy^2 + yz^2 + zx^2)
				\vect{
					(y^2+2xz),
					(2xy+z^2),
					(2yz+x^2)
				} \\
				\nabla G &= \vect{2,3,4} \\
				\nabla H &= \vect{4x^3, 0, 4z^3}
			\end{align*}
			Setting $\nabla F = \lambda\nabla G + \mu\nabla H$ gives the system to solve:
			\[
				\left\{
					\begin{aligned}
				-\sin(xy^2 + yz^2 + zx^2) (y^2+2xz) &= 2\lambda + 4\mu x^3 \\
				-\sin(xy^2 + yz^2 + zx^2) (2xy+z^2) &= 3\lambda \\
				-\sin(xy^2 + yz^2 + zx^2) (2yz+x^2) &= 4\lambda + 4\mu z^3 \\
						2x+3y+4z &= 0 \\
						x^4 + z^4 &= 625
					\end{aligned}
				\right.
			\]
		\end{solution}
	\end{parts}

	\newpage
	\question[14]
	For the function $f(x,y) = 4x^2 - 2xy - y^3$ find and classify all of the critical points.
	\begin{solution}
		\begin{align*}
			\nabla f = 0
			&\implies
			\left\{
				\begin{aligned}
					8x-2y &= 0 \\ 
					-2x -3y^2 &= 0
				\end{aligned}
			\right.
		\end{align*}
		Solving this system gives two critical points: 
		$(0,0)$ and $(-\frac1{24}, -\frac16)$.
		To classify them, the discriminant is
		\[
			f_{xx} f_{yy} - (f_{xy})^2 
			= 8(-6y) - (-2)^2 = -48y - 4
		\]
		For $(0,0)$, the discriminant gives $-4$, so $(0,0)$ is a saddle point.

		For $(-\frac1{24}, -\frac16)$, the discriminant gives 4, and $f_{xx} = 8 > 0$, so $(-\frac1{24}, -\frac16)$ is a local minimum.
	\end{solution}

	\newpage
	\question[20]
	Find the maximum and the minimum of the function
	\[
		f(x,y) = x^2 + 2x + y^2 + 6y
	\]
	in the region given by 
	\[
		g(x,y) = x^2 + y^2 \le 40.
	\]
	Show your work carefully in this problem, and let us know what you are doing.
	\begin{solution}
		First we deal with the case $g(x, y) < 40$ by solving for $\nabla f =0$:
		\[
			\vect{2x+2, 2y+6} = 0 \implies x=-1, y=-3
		\]
		We will save the evaluating for the end.

		Next we deal with the case $g(x, y) = 40$ by solving for $\nabla f = \lambda \nabla g$:
		\[
			\begin{aligned}
			&\vect{2x+2, 2y+6} = \lambda \vect{2x, 2y} \\
			& \implies \left\{
				\begin{aligned}
					2x+2 &= 2\lambda x\\
					2y+6 &= 2\lambda y
				\end{aligned}
			\right.
			\end{aligned}
		\]
		Solving the system yields the values
		\[
			x = \frac{-1}{1-\lambda} \qquad y = \frac{-3}{1-\lambda}
		\]
		which substituting into the constraint gives
		\[
			\frac{1}{(1-\lambda)^2} + \frac{9}{(1-\lambda)^2} = 40 \implies \frac{1}{1-\lambda} = \pm 2
		\]
		This gives the points $(-2, -6)$ and $(2, 6)$.
		Evaluating at all found points:
		\begin{align*}
			f(-1, -3) &= -10\\ 
			f(-2, -6) &= 0\\
			f(2, 6) &= 80 
		\end{align*}
		Hence the minimum and maximum in the given region are $-10$ and 80, respectively.
	\end{solution}

	% \begin{solution}
	% 	Use Lagrange multipliers with $g(x,y) = x^2 + y^2 = k$, for $k \in [0, 40]$.
	% 	\begin{align*}
	% 		\nabla f &= \vect{2x + 2, 2y+6}\\
	% 		\nabla g &= \vect{2x, 2y}
	% 	\end{align*}
	% 	so we solve the system of equations
	% 	\[
	% 		\left\{
	% 			\begin{aligned}
	% 				2x+2 &= 2\lambda x \\ 
	% 				2y+6 &= 2\lambda y \\ 
	% 				x^2 + y^2 &= k
	% 			\end{aligned}
	% 		\right.
	% 	\]
	% 	Solving the first two equations for $\lambda$ yields
	% 	\[
	% 		x = \frac{1}{\lambda - 1} \qquad y = \frac{3}{\lambda-1}
	% 	\]
	% 	Substituting into the third gives 
	% 	\[
	% 		\frac{10}{(\lambda-1)^2} = k \implies \frac{1}{\lambda-1} = \pm \sqrt{\frac{k}{10}}
	% 	\]
	% 	so for a fixed $k$, the optimization problem yields two solutions:
	% 	\[
	% 		(x, y) = \left( \pm \sqrt{\frac{k}{10}}, \pm 3 \sqrt{\frac{k}{10}} \right)
	% 	\]
	% 	To determine max/min values, we plug these points into $f$:
	% 	\begin{align*}
	% 		f &= x^2 + y^2 + 2x + 6y \\
	% 		&= k + 2\left(\pm \sqrt{\frac{k}{10}}\right) + 6\left( \pm 3 \sqrt{\frac{k}{10}} \right)  \\
	% 		&= k \pm 20 \sqrt{\frac{k}{10}} \\
	% 		&= k \pm 2 \sqrt{10}\sqrt{k}
	% 	\end{align*}
	% 	We want to optimize $f$ with respect to $k \in [0,40]$, which requires us to evaluate $k$ at critical points and at end points.
	% 	Note there are two cases for $f$. 
	% 	\begin{outline}
	% 		\1 If $f = k + 2 \sqrt{10} \sqrt{k}$, then $f' = 1 + \frac{\sqrt{10}}{\sqrt k} = 0$ which yields no critical points. 
	% 		\2 Evaluating at $k = 0$, $f = 0$
	% 		\2 Evaluating at $k = 40$, $f = 40 + 2 \sqrt{400} = 80$
	% 		\1 If $f = k -2 \sqrt{10} \sqrt{k}$, then $f' = 1 - \frac{\sqrt{10}}{\sqrt{k}} = 0 \implies k = 10$ is a critical point.
	% 		\2 Evaluating at $k = 0$, $f = 0$
	% 		\2 Evaluating at $k = 10$, $f = 10 - 2\sqrt{100} = -10$
	% 		\2 Evaluating at $k = 40$, $f = 40 - 2 \sqrt{400} = 0$
	% 	\end{outline}
	% 	Thus the maximum for this function in the specified region is 80 (which occurs at $(2, 6)$) and the minimum is $-10$ (which occurs at $(-1, -3)$).
	% \end{solution}

	\newpage
	\question[8]
	Suppose that $x = r \cos \theta$ and $y = r \sin \theta$ (the usual polar coordinates) and $f(x,y) = x^2 y^2$. 
	Express
	\[
		\pdv{f}{r} \qand \pdv{f}{\theta}
	\]
	as functions of $r$ and $\theta$. (Hint/Comment: Do this however you like.)
	\begin{solution}
		\begin{align*}
			f &= r^4 \cos^2\theta \sin^2\theta \\ 
			f_r &= 4r^3 \cos^2\theta \sin^2\theta \\
			f_\theta &= r^4 \left(2\cos\theta(-\sin\theta)\sin^2\theta + \cos^2\theta \cdot 2\sin\theta\cos\theta \right) \\
			&= r^4 (-2\cos\theta \sin^3\theta + 2\sin\theta \cos^3\theta)
		\end{align*}
	\end{solution}

	\newpage
	\question[18]
	Short answers \ldots
	\begin{parts}
		\part If $f$ is a function of $x$ and $y$, and $x$ and $y$ are each functions of $r$, $s$, and $t$, then use the chain rule to express $\dpd{f}{t}$.
		\begin{solution}
			\[
				\dpd ft = \dpd fx \dpd xt + \dpd fy \dpd yt
			\]
		\end{solution}

		\part Find the average value of the function $f(x,y) = xy^2$ on the rectangle $0 \le x \le 4$, $0 \le y\le 3$.
		\begin{solution}
			\[
				\frac1{12} \int_0^3 \int_0^4 xy^2 \dif x \dif y
				= \frac1{12} \left[\frac{y^3}{3}\right]_0^3 \left[\frac{x^2}{2}\right]_0^4
				= \frac1{12} (9)(8) = 6
			\]
		\end{solution}

		\part According to the theorem that we learned, what should you require of a set $S$ to guarantee that any continuous function $f$ will attain an absolute maximum and an absolute minimum on $S$?
		\begin{solution}
			The set $S$ must be closed and bounded.
		\end{solution}

		\part For the set $5x^2 + 2y^3 + 2z^6 - 3xy^2z^2 = 3$ write down the tangent plane at the point $(-1, -2, 1)$.
		\begin{solution}
			Letting $f = 5x^2 + 2y^3 + 2z^6 - 3xy^2z^2 - 3$, 
			\begin{align*}
				\nabla f &= \vect{10x - 3y^2z^2, 6y^2 - 6xyz^2, 12z^5 - 6xy^2z} \\ 
				\nabla f(-1, -2, 1) &= \vect{-22,12,36}
			\end{align*}
			\[
				\boxed{0 = -22(x+1) + 12(y+2) + 36(z-1)}
			\]
		\end{solution}
	\end{parts}
\end{questions}
\end{document}
