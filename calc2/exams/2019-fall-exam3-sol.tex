\documentclass[12pt,answers]{exam}

\usepackage{amsmath,amsfonts,amssymb,mathtools,physics,commath}
\usepackage{todonotes}
\usepackage{float}
\usepackage{multicol}
\usepackage{fancybox}
\usepackage{siunitx}
\usepackage{cancel}

\newcommand{\inv}{^{-1}}
\newcommand{\RR}{\mathbb{R}}

\pagestyle{headandfoot}
\firstpageheadrule
\runningheadrule
\firstpageheader{Math 221}{Exam 3|Solutions, Page \thepage\ of \numpages}{2019 Fall}
\runningheader{Math 221}{Exam 3|Solutions, Page \thepage\ of \numpages}{2019 Fall}
\runningfooter{}{}{}

\begin{document}
% \maketitle
\begin{questions}
\question
Determine whether the following series converge or diverge. Show all work to justify your answers.
\begin{parts}
\part[8]
$\displaystyle \sum_{n=1}^\infty \frac{3n^2-5}{n^4-n^2+1}$
\begin{solution}
  Using Limit Comparison Test with the convergent $p$-series $\sum \frac{1}{n^2}$:
  \[
  \lim_{n\to\infty} \frac{3n^2-5}{n^4-n^2+1} \cdot \frac{n^2}{1} = \lim_{n\to\infty} \frac{3n^4-5n^2}{n^4-n^2+1} = 3 < \infty
  \]
  Thus by Limit Comparison Test, the given series also \fbox{converges}
\end{solution}

\part[8]
$\displaystyle \sum_{n=1}^\infty \sqrt[n]{n}$
\begin{solution}
  This series \fbox{diverges} by divergence test, since
  \begin{align*}
    \lim_{n\to\infty} n^{\frac1n}
    &= \lim_{n\to\infty} e^{\ln n^{\frac1n}}
   = \lim_{n\to\infty} e^{\frac{\ln n}{n}}
   = \exp(\lim_{n\to\infty} \frac{\ln n}{n}) \\
   &\overset{L'H}{=}
   \exp(\lim_{n\to\infty} \frac{1}{n})
   = e^0 = 1 \ne 0
  \end{align*}
\end{solution}
\end{parts}

\newpage
\question
Determine whether the following series converge or diverge. Show all work to justify your answers.
\begin{parts}
\part[8]
$\displaystyle \sum_{n=2}^\infty \frac{1}{n\ln(n)}$
\begin{solution}
  We can use the integral test, since $\dfrac{1}{x\ln x}$ is continuous, positive, and decreasing on the interval $x > 2$. Using $u = \ln x$,
  \[
  \int_2^\infty \frac{1}{x\ln x}\dif x
  = \int \frac{1}{u} \dif u
  = \ln |u|
  = \eval{\ln\abs{\ln x}}_2^\infty = \infty
  \]
  Thus by the integral test, the series \fbox{diverges}
\end{solution}

\part[8]
$\displaystyle \sum_{n=1}^\infty \frac{3^n}{(n+1)^n}$
\begin{solution}
  By the root test:
  \[
  \lim_{n\to\infty} \sqrt[n]{\left(\frac{3}{n+1}\right)^n} = \lim_{n\to\infty} \frac{3}{n+1} = 0 < 1
  \]
  the series \fbox{converges}
\end{solution}
\end{parts}

\newpage
\question
\begin{parts}
\part[5]
Show the following series converges
\[
\sum_{n=1}^\infty \frac{(-1)^n}{3n+5}
\]
\begin{solution}
  To use the Alternating Series Test, we check that $b_n = \dfrac{1}{3n+5}$ satisfies the conditions:
  \begin{itemize}
    \item Decreasing ($b_{n+1}<b_n$): $\dfrac{1}{3(n+1)+5} < \dfrac{1}{3n+5}$ \checkmark
    \item $b_n \to 0$: $\displaystyle \lim_{n\to\infty} \dfrac{1}{3n+5} = 0$ \checkmark
  \end{itemize}
Thus by the Alternating Series Test, this series converges.
\end{solution}

\part[6]
Find the minimum $M$ that guarantees
\[
\left|\sum_{n=1}^\infty \frac{(-1)^n}{3n+5} - \sum_{n=1}^M \frac{(-1)^n}{3n+5} \right| < 0.01
\]
\begin{solution}
  \begin{align*}
    \frac{1}{3(M+1)+5} < \frac{1}{100}
    &\implies 100 < 3(M+1)+5 \\
    &\implies \frac{95}{3} < M+1 \\
    &\implies M > \frac{92}{3} = 30.\overline{6}
  \end{align*}
So \fbox{$M=31$} suffices.
\end{solution}
\end{parts}

\newpage
\question
Determine whether the following series converge conditionally, converge absolutely, or diverge. Justify your answer.
\begin{parts}
  \part[8]
  $\displaystyle \sum_{n=1}^\infty \frac{(-1)^n\cos(n)}{2^n}$
  \begin{solution}
    Since $\dfrac{\abs{\cos(n)}}{2^n} < \dfrac{1}{2^n}$ and $\displaystyle \sum \dfrac{1}{2^n}$ converges (geometric series with $r = \frac12 < 1$), 
    $\displaystyle \sum \frac{|\cos(n)|}{2^n}$ converges by direct comparison test. 
    Thus the given series \fbox{converges absolutely}
  \end{solution}

  \part[8]
  $\displaystyle \sum_{n=1}^\infty \frac{(-1)^n n}{n+1}$
  \begin{solution}
    Note that 
    \[
      \lim_{n\to\infty} \frac{n}{n+1} = 1
    \]
    so 
    \[
      \lim_{n\to\infty} \frac{(-1)^n n}{n+1}  \quad \text{ DNE}
    \]
    Thus by the divergence test, the series \fbox{diverges}
  \end{solution}
\end{parts}

\newpage
\question
Determine for which values of $x$ the following power series converge.
\begin{parts}
  \part[8]
  $\displaystyle \sum_{n=1}^\infty \frac{5x^2}{(n+5)n!}$
  \begin{solution}
    \begin{align*}
      \lim_{n\to\infty} \left| \frac{5 x^{n+1}}{(n+6)(n+1)!} \cdot \frac{(n+5)n!}{5x^n}\right|
      &= \lim_{n\to\infty} \left| x \frac{n+5}{(n+6)(n+1)}\right| \\
      &= |x| \lim_{n\to\infty} \frac{n+5}{(n+6)(n+1)} = 0
    \end{align*}
    Thus the power series converges for all \fbox{$x \in \RR$}
  \end{solution}
  \part[8]
  $\displaystyle \sum_{n=1}^\infty \frac{n^2(x-1)^n}{3^n}$
  \begin{solution}
    \begin{align*}
      \lim_{n\to\infty} \abs{ \frac{(n+1)^2(x-1)^{n+1}}{3^{n+1}} \cdot \frac{3^n}{n^2(x-1)^n}}
      &= \lim_{n\to\infty} \abs{\frac{(n+1)^2}{n^2} \cdot \frac{x-1}{3}} \\
      &= \frac13 \abs{x-1} \lim_{n\to\infty} \frac{(n+1)^2}{n^2} \\
      &= \frac13 \abs{x-1} < 1 
      \implies \abs{x-1} < 3
    \end{align*}
    At $x = 4$: $\displaystyle \sum_{n=1}^\infty n^2$ diverges \\ 
    At $x = -2$: $\displaystyle \sum_{n=1}^\infty (-1)^n n^2$ diverges \\
  Thus the power series converges for \fbox{$x \in (-2, 4)$}
  \end{solution}
\end{parts}

\newpage
\question
\begin{parts}
  \part[8]
Find the power series for the function $f(x) = \dfrac{2}{1+x^3}$.
Determine the interval of convergence of the series.
\begin{solution}
  \begin{align*}
    \frac{1}{1-x} &= \sum_{n=0}^\infty x^n \qquad (|x| < 1) \\ 
    \frac{2}{1+x^3} &= \frac{2}{1-(-x^3)} = 2 \sum_{n=0}^\infty (-x^3)^n \qquad (|-x^3|<1 \implies |x| < 1) \\ 
    &= 2 \sum_{n=0}^\infty (-1)^n x^{3n} \qquad (|x| < 1)
  \end{align*}
  The interval of convergence of this series is $\boxed{(-1, 1)}$
\end{solution}
  \part[8]
Consider the power series $g(x) = \displaystyle \sum_{n=1}^\infty \frac{x^n}{n}$.
Find the power series for $\displaystyle \int \frac{g(x)}{x} \dif x$.
\begin{solution}
  \begin{align*}
    \frac{g(x)}{x} &= \sum_{n=1}^\infty \frac{x^{n-1}}{n} \\ 
    \int \frac{g(x)}{x} \dif x &= \boxed{\sum_{n=1}^\infty \frac{x^{n}}{n^2}}
  \end{align*}
\end{solution}
\end{parts}

\newpage
\question[9]
Find the degree three Taylor polynomial of
\[
f(x) = \ln(1-x)
\]
centered at $x=0$.
\begin{solution}
\begin{alignat*}{2}
  f(x)    & = \ln(1-x)       & f(0)           & = 0  \\
  f'(x)   & = -\frac{1}{1-x} & f'(0)          & = -1 \\
  f''(x)  & = -(1-x)^{-2}    & f''(0)         & = -1 \\
  f'''(x) & = -2(1-x)^{-3}   & \qquad f'''(0) & = -2
\end{alignat*}
so
\[
T_3(x) = 0 + (-1)x + (\frac{-1}{2!})x^2 + (\frac{-2}{3!})x^3 =
\boxed{-x -\frac12x^2 - \frac13x^3}
\]
\end{solution}
\end{questions}
\end{document}