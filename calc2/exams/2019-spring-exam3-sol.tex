\documentclass[12pt,answers]{exam}

\usepackage{amsmath,amsfonts,amssymb,mathtools,physics,commath}
\usepackage{todonotes}
\usepackage{float}
\usepackage{multicol}
\usepackage{polynom}
\usepackage{siunitx}
\usepackage{cancel}

\newcommand{\inv}{^{-1}}

\pagestyle{headandfoot}
\firstpageheadrule
\runningheadrule
\firstpageheader{Math 221}{Exam 3|Solutions, Page \thepage\ of \numpages}{April 9, 2019}
\runningheader{Math 221}{Exam 3|Solutions, Page \thepage\ of \numpages}{April 9, 2019}
\runningfooter{}{}{}

\begin{document}
% \maketitle
\begin{questions}

\question
Evaluate the following.
\begin{parts}
    \part[8]
    $\displaystyle \dod{}{x} e^{2x} \sinh\inv(\sqrt x)$, where $\sinh\inv$ is the inverse sinh function.
    \begin{solution}
        \begin{align*}
            2e^{2x} \sinh\inv(\sqrt x) + e^{2x} \frac{1}{\sqrt{1+x}} \cdot \frac{1}{2\sqrt x}
        \end{align*}
    \end{solution}
    \part[8]
    $\displaystyle \int \sqrt{1+x^2} \dif x$, using the substitution $x = \sinh(\theta)$.
    \begin{solution}
        \begin{align*}
            \MoveEqLeft \int \sqrt{1+x^2} \dif x \\ 
             & = \int \sqrt{1+\sinh^2(\theta)} \cosh(\theta) \dif \theta                                                    \\
             & = \int \sqrt{\cosh^2(\theta)} \cosh(\theta) \dif \theta                                                      \\
             & = \int \cosh^2(\theta) \dif \theta                          & (\cosh^2 \theta = \frac12(1+\cosh(2\theta)) )  \\
             & = \int \frac12 (1+\cosh(2\theta)) \dif \theta                                                                \\
             & = \frac12 \theta + \frac14 \sinh(2\theta)                   & (\sinh(2\theta) = 2\sinh(\theta)\cosh(\theta)) \\
             & = \frac12 \theta + \frac12 \sinh(\theta) \cosh(\theta)      & (\cosh^2 \theta - \sinh^2 \theta = 1)          \\
             & = \boxed{\frac12 \sinh\inv(x) + \frac12 x \sqrt{1+x^2} + C}
        \end{align*}
    \end{solution}
\end{parts}

\newpage
\question
Consider the differential equation
\[
    \dod{y}{x} = \frac{\ln(x)}{xy^2}, \qquad (x > 0).
\]
\begin{parts}
    \part[8]
    Find the general solution.
    \begin{solution}
        \begin{align*}
            \int y^2 \dif y       & = \int \frac{\ln(x)}{x} \dif x       \\
            \implies \frac13 y^3  & = \frac 12 (\ln(x))^2 + C_1          \\
            \implies y^3          & = \frac 32 (\ln(x))^2 + C_2          \\
            \implies \Aboxed{y(x) & = \sqrt[3]{\frac 32 (\ln(x))^2 + C}}
        \end{align*}
    \end{solution}
    \part[2]
    Find the solution satisfying the initial condition $y(1) = 4$.
    \begin{solution}
        \begin{align*}
            y(1)         & = \sqrt[3]{\frac32 \cdot 0 + C} = 4
            \implies C = 4^3 = 64                                \\
            \Aboxed{y(x) & = \sqrt[3]{\frac 32 (\ln(x))^2 + 64}}
        \end{align*}
    \end{solution}
\end{parts}

\newpage
\question
Find the limit of the sequence or state that it diverges.
\begin{parts}
    \part[6]
    $\displaystyle \lim_{n\to\infty} \frac{(\ln n)^2}{n}$
    \begin{solution}
        \begin{align*}
            \lim_{n\to\infty} \frac{(\ln n)^2}{n}
            \overset{LH}{=}
            \lim_{n\to\infty} \frac{2\ln n \cdot \frac1n}{1}
            =
            \lim_{n\to\infty} \frac{2\ln n}{n}
            \overset{LH}{=}
            \lim_{n\to\infty} \frac{2 \cdot \frac1n}{1}
            = \boxed{0}
        \end{align*}
    \end{solution}
    \part[6]
    $\displaystyle \lim_{n\to\infty} n \sin(2/n)$
    \begin{solution}
        This has type $\infty \cdot 0$.
        \begin{align*}
            \lim_{n\to\infty} n \sin(2/n)
            =
            \lim_{n\to\infty} \frac{\sin(2/n)}{n\inv}
             & \overset{LH}{=}
            \lim_{n\to\infty} \frac{\cos(2/n) \cdot -2n^{-2}}{-n^{-2}} \\
             & =
            \lim_{n\to\infty} 2\cos(2/n)
            = \boxed{2}
        \end{align*}
    \end{solution}
\end{parts}

\question
Evaluate the series.
\begin{parts}
    \part[6]
    $\displaystyle \sum_{n=2}^\infty (-1)^n \left(\frac23\right)^n$
    \begin{solution}
        Geometric series with $r = -\frac23$, and initial term $a = (-1)^2 \left(\frac23\right)^2 = \frac49$.
        The sum is thus
        \[
            \frac{a}{1-r}
            = \frac{\frac49}{1-(-\frac23)}
            = \frac49 \cdot \frac 35
            = \boxed{\frac{4}{15}}
        \]
    \end{solution}
    \part[8]
    $\displaystyle \sum_{n=3}^\infty \frac{2}{n^2-1}$
    \begin{solution}
        \begin{align*}
            \frac{2}{n^2-1}
             & = \frac{1}{n-1} - \frac{1}{n+1}                              \\
            \sum_{n=3}^\infty \frac{2}{n^2-1}
             & =
            \sum_{n=3}^\infty \left[ \frac{1}{n-1} - \frac{1}{n+1}  \right] \\
             & = \left( \frac12 - \cancel{\frac 14} \right)
            + \left( \frac13 - \cancel{\frac 15} \right)
            + \left( \cancel{\frac 14} - \cancel{\frac 16} \right)
            + \cdots                                                        \\
             & = \frac12 + \frac 13 = \boxed{\frac 56}
        \end{align*}
    \end{solution}
\end{parts}

% \newpage
\question[7]
Use the limit comparison test to determine whether the following series converges or diverges. 
Show all work to justify your answer.
$\displaystyle \sum_{n=1}^\infty \frac{n^2+8}{4n^4-n^2}$
\begin{solution}
    Compare against $\displaystyle \sum_{n=1}^\infty \frac{1}{n^2}$.
    \[
        \lim_{n\to\infty} \frac{\dfrac{n^2+8}{4n^4-n^2}}{\dfrac{1}{n^2}}
        = 
        \lim_{n\to\infty} \frac{n^4+8n^2}{4n^4-n^2} 
        = \frac14
    \]
    and $\displaystyle \sum_{n=1}^\infty \frac{1}{n^2}$ converges by $p$-series test. 
    So by Limit Comparison Test, \\
    $\displaystyle \sum_{n=1}^\infty \frac{n^2+8}{4n^4-n^2}$ \fbox{converges}
\end{solution}

% \newpage
\question
Determine whether the following series converge or diverge. Show all work to justify your answers.
\begin{parts}
    \part[7]
    $\displaystyle \sum_{n=1}^\infty \frac{1}{n^{3/2}}$
    \begin{solution}
        \fbox{Converges} by $p$-series test, with $p = 3/2 > 1$.
    \end{solution}
    \part[7]
    $\displaystyle \sum_{n=1}^\infty \cos(1/n)$
    \begin{solution}
        \fbox{Diverges} by divergence test since $\displaystyle \lim_{n\to\infty} \cos(1/n) = 1 \ne 0$.
    \end{solution}
    \part[7]
    $\displaystyle \sum_{n=2}^\infty \frac{1}{n\ln(n)}$
    \begin{solution}
        \fbox{Diverges} by integral test, since 
        $f(x) = \dfrac{1}{x \ln x}$ is a positive, continuous, decreasing function on $(2, \infty)$
        and 
        \[
            \int_2^\infty \frac{1}{x \ln x} \dif x = \eval{\ln\abs{\ln x}}_2^\infty \quad \text{ diverges}
        \]
    \end{solution}
\end{parts}

\newpage
\question[6]
Find the minimum $M$ that guarantees that 
\[
    \left| \sum_{n=1}^\infty \frac{(-1)^{n+1}}{2n-1} - \sum_{n=1}^M \frac{(-1)^{n+1}}{2n-1} \right| < 0.01.
\]
\begin{solution}
$ \left| \sum_{n=1}^\infty \frac{(-1)^{n+1}}{2n-1} - \sum_{n=1}^M \frac{(-1)^{n+1}}{2n-1} \right| \le \frac{1}{2(M+1)-1}$.
We want
% < 0.01 = \frac{1}{100}$ so
    \begin{align*}
    \frac{1}{2(M+1)-1} < 0.01 = \frac{1}{100}
    &\implies 100 < 2(M+1) - 1 \\ 
    &\implies \frac{101}{2} - 1 < M \\
    &\implies M > 49.5
    \end{align*}
    So $\boxed{M = 50}$ works.
\end{solution}

\question
Determine whether the following series converge conditionally, converge absolutely, or diverge. Justify your answer. 
\begin{parts}
    \part[7]
    $\displaystyle \sum_{n=1}^\infty \frac{(-1)^n}{\sqrt n}$
    \begin{solution}
        $\displaystyle \sum_{n=1}^\infty \frac{1}{\sqrt{n}}$ diverges by $p$-series test ($p = \frac12 < 1$).
       
        $\displaystyle \lim_{n\to\infty} \frac{1}{\sqrt n} = 0$ and $\displaystyle \frac{1}{\sqrt{n+1}} \le \frac{1}{\sqrt{n}}$ for $n \ge 1$, 
        so by the Alternating Series Test, $\displaystyle \sum_{n=1}^\infty \frac{(-1)^n}{\sqrt n}$ converges.

        Thus
        $\displaystyle \sum_{n=1}^\infty \frac{(-1)^n}{\sqrt n}$ \fbox{converges conditionally}
    \end{solution}
    \part[7]
    $\displaystyle \sum_{n=1}^\infty \frac{(-1)^{n+1} \sin(n)}{5e^n}$
    \begin{solution}
        The series \fbox{converges absolutely} since 
        \[
            \sum_{n=1}^\infty \frac{\sin(n)}{5e^n} 
            \le 
            \sum_{n=1}^\infty \frac{1}{5e^n} 
        \]
        and the right series converges, being a geometric series with $r = \frac1e \approx \frac{1}{2.7} < 1$.
    \end{solution}
\end{parts}
\end{questions}
\end{document}